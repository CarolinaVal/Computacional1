\documentclass[12 pt,twocolumn]{article}
\usepackage[utf8]{inputenc}
\usepackage[spanish,mexico]{babel}
\usepackage{amsmath}
\usepackage{amssymb}
\usepackage{graphicx}
\usepackage{amsfonts}
\usepackage{float}
\usepackage{moreverb}
\begin{document}
\title{Actividad 2}
\author{Carolina Valenzuela Córdova}
\date{28 de Enero de 2016}
\maketitle
\newpage

En esta actividad, se nos proporcionaron modelos de códigos en Python para que realizaran ciertas operaciones matemáticas y nos dieran resultados diferentes. Utilizamos una plataforma en línea para poder programar con Python y correr los programas que modificamos. En la actividad se indicaron 5 programas diferentes y a continuación se brinda su descripción y las capturas de pantalla pertinentes.
\section{\small Problema 1: Caída libre}
Se nos proporcionó un código que calcula la distancia que recorre una pelota cuando se deja caer desde el techo de una torre, es decir, calcula la altura de la misma.
Las instrucción era modificarlo para que solicitara al usuario la altura de la torre en metros, y entonces que imprimiera el tiempo que tarda la pelota en llegar al suelo, ignorando la fricción del aire. 
El código original era el siguiente:
\begin{center}
\begin{boxedverbatim}
h = float(input("Proporciona la 
altura
de la torre: "))
t = float(input("Ingresa el
tiempo:"))
s = 0.5*9.81*t**2
print("La altura de la pelota es",
h-s,
"metros")
\end{boxedverbatim}
\end{center}
Se tuvo que reemplazar la ecuación utilizada en el código original por una que calculara el tiempo de caída de la pelota:$$t=\sqrt{\frac{2h}{g}}$$, siendo $g=9.81 m/s^2$.
También cambiamos el hecho de que el programa solicitara el tiempo, siendo ahora solo requerida la altura de la torre. El programa ya modificado resultó así:
\begin{center}
\begin{boxedverbatim}
h = float(input("Proporciona la altura
de la torre: "))
t = sqrt((2*h)/9.81)
print ("El tiempo que tarda la pelota
en caer son",t,"segundos")
\end{boxedverbatim}
\end{center}
y podemos observar los resultados:
\begin{verbatim}
Proporciona la altura de la torre: 20
('El tiempo que tarda la pelota en caer 
son', 2.0192751093846089, 'segundos')
\end{verbatim}



\section{\small Problema 2: Satélite}
Se nos solicitó elaborar un programa que calculara las altitudes para que un satélite orbitara la Tierra una vez al día, cada 90 y 45 minutos, a partir de pedirle al usuario que ingresara el valor de $T$, de tal manera que le regresara un una altura $h$ en metros.
Se nos proporcionó la siguiente ecuación: $$(R+h)^3=\frac{GMT^2}{4\pi^2}$$ de la cual, despejando $h$ obtuvimos $$h=\sqrt[3]{\frac{GMT^2}{4\pi^2}} -R $$ Esta fue la ecuación que insertamos en el código. Además, tuvimos que realizar algunas conversiones para que todas las unidades coincidieran, teniendo como valores constantes $G=6.67x10^-11 m^3  kg^-1  s^-1$,(constante de Gravitación Universal de Newton), $M=5.97x10^24 kg$(masa de la Tierra) y $R=6371000 m$ (radio de la Tierra), además, el periodo que ingresa el usuario se solicita en segundos.
\begin{center}
\begin{boxedverbatim}
from math import pi
T= float(input("Ingrese un valor para 
el periodo en segundos"))
G=6.67e-11
M=5.97e24
R=6371000
h=(((G*M*T*T)/(4*pi*pi))**(1./3.))-R
print ("El valor de la altitud del 
satelite sobre la superficie de la 
Tierra es",h, "metros")
\end{boxedverbatim}
\end{center}
Aquí se muestran los resultados obtenidos para 45 y 90 minutos, es decir 2700 y 5400 segundos, respectivamente.
\begin{verbatim}
Ingrese un valor para el periodo en
segundos2700
('El valor de la altitud del satelite 
sobre la superficie de la Tierra es',
-2181559.8978108233, 'metros')

Ingrese un valor para el periodo en 
segundos5400
('El valor de la altitud del satelite
sobre la superficie de la Tierra es',
279321.6253728606, 'metros')

\end{verbatim}

\section{\small Problema 3: Coordenadas polares}
Se proporcionó un código para transformar coordenadas cartesianas a partir de coordenadas polares:
\begin{center}
\begin{boxedverbatim}
from math import sin,cos,pi
r = float(input("Introduce r: "))
d = float(input("Ingresa theta en 
grados: "))
theta = d*pi/180
x = r*cos(theta)
y = r*sin(theta)
print("x =",x," y =",y)
\end{boxedverbatim}
\end{center}
Se solicitó hacer un código similar para calcular coordenadas esféricas a partir de cartesianas. Para ello fue necesario encontrar una relación entre estos dos tipos de coordenadas y esta fue $$r=\sqrt{x^2+y^2+z^2}$$ $$\theta=\arccos{\frac{z}{\sqrt{x^2+y^2+z^2}}}$$ $$\varphi=\arctan{\frac{y}{x}} $$
Sabiendo esto, pudimos escribir un código que realizara estas operaciones solicitándole al usuario que ingresara valores para $x$, $y$ y $z$.
Nuestro código resultó de la siguiente manera:
\begin{center}
\begin{boxedverbatim}
from math import sin,cos,pi,
acos,atan
x = float(input("Introduce 
un valor 
para x: "))
y = float(input("Introduce
un valor 
para y:"))
z = float(input("Introduce 
un valor 
para z:"))
r=sqrt((x**2)+(y**2)+(z**2))
theta=acos((z/r))
phi=atan((y/x))

print("r=",r," theta =",theta,"phi="
,phi)
\end{boxedverbatim}
\end{center}
Resultados obtenidos al correr el programa:
\begin{verbatim}
Introduce un valor para x: 1
Introduce un valor para y:2
Introduce un valor para z:1
('r=', 2.4494897427831779, ' theta =', 
1.1502619915109313, 'phi=',
1.1071487177940904)
\end{verbatim}

\section{\small Problema 4: Números pares e impares}
En este ejercicio solo se solicitó correr un programa que detecta números pares e impares y comprender su dinámica, pues utiliza el comando $while$.
El programa es el siguiente:
\begin{center}
\begin{boxedverbatim}
print("Ingrese dos números, 
un par y un 
impar.")
m = int(input("Ingrese 
el primero: "))
n = int(input("Ingrese
el segundo: "))
while (m+n)%2==0:
    print("Uno debe ser 
    par y otro 
    impar.")
    m = int(input("Ingrese
    el primer
    número: "))
    n = int(input("Ingrese
    el segundo 
    número: "))
print("Los numeros que 
ligio son ",m,"y",n)
\end{boxedverbatim}
\end{center}
Pudimos observar que el comando se utiliza para realizar ciclos, los cuales sirven para resolver ecuaciones en repetidas ocasiones con diferentes valores que se establecen en el mismo programa, a esto comúnmente se le llama $iterar$.
La comprensión del comando nos permitió realizar con mayor facilidad el siguiente programa.

\section{\small  Problema 5: Secuencia de números de Catalán}
Se nos solicitó escribir un código que calculara la secuencia de números de Catalán menores o igual que $1,000,000$.Fue más sencillo realizar este código porque ya habíamos comprendido el uso del comando $while$ y además nos dieron un ejemplo similar que concierne a la sucesión de números Fibonacci.
El código que usamos como ejemplo fue:
\begin{center}
\begin{boxedverbatim}
f1,f2 = 1,1
while f2<1000:
      print(f2)
      f1,f2 = f2,f1+f2
\end{boxedverbatim}
\end{center}
Se nos proporcionó de igual manera una ecuación que calculara la sucesión de números de Catalá, la cual es $$C_{n+1}=\frac{2(2n+1)}{n+2}C_n$$
Así, con conocimiento de estos datos pudimos realizar el programa, el cual resultó ser:
\begin{center}
\begin{boxedverbatim}
 n,c=0.,1.
while c<=1000000:
    print(c)
    n,c = (n+1),(2*(2*n+1)
    /(n+2))*c
 \end{boxedverbatim}
 \end{center}
Resultados obtenidos al correr el programa:
\begin{verbatim}
1.0
1.0
2.0
5.0
14.0
42.0
132.0
429.0
1430.0
4862.0
16796.0
58786.0
208012.0
742900.0
\end{verbatim}

\section{\small Conclusión}
Me pareció muy importante la realización de esta práctica, pues pudimos comenzar a tener contacto con diferentes tipos de programas que se pueden realizar con Python. Además pudimos comprenderel funcionamiento de algunos comandos nuevos y la estructura de los códigos en este tipo de compliador. Podemos notar de igual manera que nuestra experiencia con programación nos ha ayudado a comprender con mayor facilidad los procedimientos para crear códigos, así como para hacer reportes con mayor calidad, gracias a la práctica anterior.




\end{document}