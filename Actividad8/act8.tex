\documentclass[12 pt,twocolumn]{article}
\usepackage[utf8]{inputenc}
\usepackage[spanish,mexico]{babel}
\usepackage{amsmath}
\usepackage{amssymb}
\usepackage{graphicx}
\usepackage{amsfonts}
\usepackage{float}

\begin{document}
\title{Actividad 8: Bitácora de funcionamiento de Maxima}
\author{Carolina Valenzuela Córdova}
\date{5 de Abril de 2016}
\maketitle
\pagebreak

\pagenumbering{gobble} % Remove page numbers (and reset to 1)
\maketitle

\newpage
\pagenumbering{arabic}
\tableofcontents
\pagebreak

\section{Geometría Tridimensional}
\subsection{Álgebra lineal y vectores}
En esta parte introdujimos vectores e hicimos algunas operaciones con ellos, tales como suma, resta, producto punto, producto cruz, etcétera.\\ A continuación se muestran los resultados de esta sección:

\begin{verbatim}
%i1) a:[1,2,3];
(%o1)                              [1, 2, 3]
(%i2) b:[2,-1,4];
(%o2)                             [2, - 1, 4]
(%i3) a+b;
(%o3)                              [3, 1, 7]
(%i4) a.b;
(%o4)                                 12
(%i5) load(vect);
(%o5)           /usr/share/maxima/5.21.1/share
/vector/vect.mac
(%i6) a~b;
(%o6)                      
 [1, 2, 3] ~ [2, - 1, 4]
(%i7) express(a~b);
(%o7)                            [11, 2, - 5]
(%i8) sqrt(a.a);
(%o8)                              sqrt(14)
(%i9) 
\end{verbatim}	

\subsection{Figuras en el espacio}
Aquí graficamos un hiperboloide similar al de un ejemplo sugerido en el manual. Para ello utilizamos los siguientes comandos, cambiando el color del mismo y los límites de graficación:
\begin{verbatim}
hyperboloid:x**2+y**2-z**2=1;

load(draw);

draw3d(enhanced3d= true,
 implicit(hyperboloid, x,-2,2,
  y,-2,2, z,-1.5,1.5)
 ,palette=[29,9,20]);
\end{verbatim}





\end{document}
